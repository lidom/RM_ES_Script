\documentclass{beamer}
\usepackage[utf8]{inputenc}
\usepackage[english]{babel}
\usepackage{amsthm}
\usepackage{amssymb}
\usepackage{bm}
\usepackage{xfrac}

\setbeamertemplate{navigation symbols}{}
\setbeamertemplate{footline}{
    \hbox{%
    \begin{beamercolorbox}[wd=\paperwidth,ht=3ex,dp=1.5ex,leftskip=2ex,rightskip=2ex]{page footer}%
        \usebeamerfont{title in head/foot}%
           \large \insertsection \hfill
        \large \insertframenumber{} / \large \inserttotalframenumber
    \end{beamercolorbox}}%
}


\title{Unbiasedness of $s^2$}
\author{Linda Maokomatanda, Tim Mensinger, Markus Schick}
\institute{University of Bonn}
\date{4. November 2019}


\begin{document}

\begin{frame}[noframenumbering,plain]
    \titlepage
\end{frame}

\begin{frame}{Content}
\fontsize{15pt}{10}\selectfont
\begin{enumerate}
    \item Introduction (Linda)
        \begin{itemize}[]
\fontsize{14pt}{7.2}\selectfont
            \item[] Assumptions 
            \item[] Statement of Theorem
        \end{itemize}
    \item Proof (Markus)
    \item Wrap-up (Tim)
\end{enumerate}
\end{frame}

\section{Introduction}
\subsection{Assumptions}

\begin{frame}{Assumption 1 - Linearity}
\fontsize{15pt}{10}\selectfont
    \begin{align*}
    \bm{y} = \mathbf{X} \beta + \bm{\varepsilon}
    \end{align*}
\end{frame}

\begin{frame}{Assumption 2 - Strict Exogeneity}
\fontsize{15pt}{10}\selectfont
\begin{align*}
    \mathbb{E}\left [\bm{\varepsilon} \mid \mathbf{X} \right] = \mathbf{0}
\end{align*}
\end{frame}

\begin{frame}{Assumption 3 - Rank Condition}

\fontsize{15pt}{10}\selectfont
\begin{align*}
    \text{rank}(\mathbf{X})=K\quad\text{a.s.}
\end{align*}
\end{frame}

\begin{frame}{Assumption 4 - Spherical Error}
\fontsize{15pt}{10}\selectfont
    \begin{align*}
     \mathbb{E}\left[\bm{\varepsilon} \bm{\varepsilon}' \right] = \sigma^2 \, \mathbf{id}_n
    \end{align*}
    \begin{center}
        for some \emph{fixed} $\sigma > 0$ 
    \end{center}
\end{frame}

\subsection{Theorem}
\begin{frame}{Statement of Theorem}
\fontsize{15pt}{10}\selectfont
    \begin{theorem}
    \vspace{5pt}
    Under assumptions 1-4, we have that
    \begin{align*}
        \mathbb{E} \left[ s^2 \mid \mathbf{X} \right] = \sigma^2 
    \end{align*}
    where $s^2 = \bm{\hat{\varepsilon}}' \bm{\hat{\varepsilon}} \sfrac{}{(n-K)}$
    \end{theorem}
\end{frame}


\section{Proof}

\begin{frame}{Proof}
\fontsize{15pt}{10}\selectfont
WTS: 
\begin{align*}
\mathbb{E} \left [ s^2 \mid \mathbf{X} \right] = \mathbb{E} \left [ \bm{\hat{\varepsilon}}' \bm{\hat{\varepsilon}} \sfrac{}{(n-K)} \mid \mathbf{X} \right] = \sigma^2
\end{align*}
\vspace{5pt}
Strategy:
\begin{enumerate}
    \item Show $\bm{\hat{\varepsilon}}' \bm{\hat{\varepsilon}} = \bm{\varepsilon}'\mathbf{M} \bm{\varepsilon}$
    \item Show $\mathbb{E} \left[ \bm{\varepsilon}'\mathbf{M} \bm{\varepsilon} \mid \mathbf{X} \right] = \sigma^2 \, \text{trace}(\mathbf{M})$
    \item Show $\text{trace}(\mathbf{M})=n-K$
\end{enumerate}
\end{frame}

\begin{frame}{1.}
\fontsize{15pt}{10}\selectfont

Remember that $\bm{\hat{\varepsilon}} = \mathbf{M} \bm{y}$. Using $\bm{y} = \mathbf{X} \beta + \bm{\varepsilon}$ and Lemma 3.1.1 it follows trivially that  
\begin{align*}
\hat{\boldsymbol{\varepsilon}}'\hat{\boldsymbol{\varepsilon}}
& = (\mathbf{M}\mathbf{y})'\mathbf{M}\mathbf{y}\\
& = (\mathbf{M}(\mathbf{X}\boldsymbol{\beta}+\boldsymbol{\varepsilon}))'\mathbf{M}(\mathbf{X}\boldsymbol{\beta}+\boldsymbol{\varepsilon})\\
& = (\mathbf{M}\boldsymbol{\varepsilon})'\mathbf{M}\boldsymbol{\varepsilon}\\
& = \boldsymbol{\varepsilon}'\mathbf{M}\boldsymbol{\varepsilon}
\end{align*}
\end{frame}

\begin{frame}{2.}
\fontsize{15pt}{10}\selectfont
Note that $\mathbf{M} = \mathbf{M}(\mathbf{X})$ and 
$\bm{\varepsilon}'\mathbf{M}\bm{\varepsilon}
=\sum_{i,j=1}^n m_{ij} \varepsilon_i\varepsilon_j$. Therefore
\begin{align*}
\mathbb{E} \left[ \bm{\varepsilon}'\mathbf{M}\bm{\varepsilon} \mid \mathbf{X} \right]
&= \sum_{i,j=1}^n m_{ij}\mathbb{E}\left[ \varepsilon_i\varepsilon_j \mid \mathbf{X} \right]\\
&= \sum_{i=1}^n m_{ii}\sigma^2 \\
&= \sigma^2 \, \text{trace}(\mathbf{M})
\end{align*}
\end{frame}

\begin{frame}{3.}
\fontsize{15pt}{10}\selectfont
\begin{align*}
\text{trace}(\mathbf{M})
&= \text{trace}(\mathbf{id}_n-\bm{P}) \\
&= \text{trace}(\mathbf{id}_n)-trace(\bm{P}) \\ 
&= n-\text{trace}(\bm{P}) \\
&= n-\text{trace}(\mathbf{X}(\mathbf{X}'\mathbf{X})^{-1}\mathbf{X}') \\
&= n-\text{trace}(\mathbf{X}'\mathbf{X}(\mathbf{X}'\mathbf{X})^{-1}) \\
&= n-\text{trace}(\mathbf{id}_K) \\ 
&= n-K
\end{align*}
\end{frame}

\begin{frame}{Proof}
\fontsize{15pt}{10}\selectfont
Combining 1 - 3 we get 
\begin{align*}
    \mathbb{E} \left [ s^2 \mid \mathbf{X} \right] 
&= \mathbb{E} \left [ \bm{\hat{\varepsilon}}' \bm{\hat{\varepsilon}} \sfrac{}{(n-K)} \mid \mathbf{X} \right] \\
&=  \mathbb{E} \left [\bm{\varepsilon}'\mathbf{M} \bm{\varepsilon} \mid \mathbf{X} \right] \sfrac{}{(n-K)} \\
&= \sigma^2 \, \text{trace}(\mathbf{M}) \sfrac{}{(n-K)} \\
&= \sigma^2
\end{align*}
which was what we wanted
\end{frame}

\section{Wrap-Up}
\begin{frame}{Wrap-Up}
\fontsize{15pt}{10}\selectfont
\begin{center}
    \emph{strong} assumptions $\implies$ \emph{weak} results
\end{center}
\end{frame}


\end{document}
