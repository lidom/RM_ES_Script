\documentclass{beamer}
\usepackage[utf8]{inputenc}
\usepackage{amsmath}
\usepackage{amsfonts}
\usepackage{amssymb}
\usepackage{mathrsfs}
\usetheme{Dresden}
\expandafter\def\expandafter\insertshorttitle\expandafter{%
   \insertshorttitle\hfill%
   \insertframenumber\,/\,\inserttotalframenumber}

\title{}
\title{$R^{2}$ increase}
\subtitle{Proposition 3.1.4}
\author{Anastasiia Gordienko, Patrick Pöpperling}
\institute{Research Module - Econometrics and Statistics}
\begin{document}
\frame{
	\titlepage 
}
\frame{
\frametitle{Content}
\begin{enumerate}
\item Reminder: What is $R^2$ and why could it be problematic?
\item Proofs
\begin{enumerate}
	\item $R^2$ increase
	\item Adjusted $R^2$
\end{enumerate}
\item Conclusion
\end{enumerate}
}

\frame{
\frametitle{What is $R^2$?}
\begin{itemize}
\item statistical measure that \textit{represents the proportion of variance} for a dependent variable \textit{that can be explained} by an independent variable
\item used to express how well a model fits the observed data
\end{itemize}
}

\frame{
\frametitle{What is $R^2$?}

\begin{align*}
R^2= \frac{\overbrace{\sum_{i=1}^{n}(\hat{y}_i-\bar{\hat{y}})^2}^{\text{Explained Variation}}}{\underbrace{\sum_{i=1}^{n}(y_i-\bar{y})^2}_{\text{Total Variation}}} = 1 - \frac{\overbrace{\sum_{i=1}^{n}\hat{\epsilon}_i^2}^{\text{Unexplained Variation}}}{\sum\limits_{i=1}^{n}(y_i-\bar{y})^2}
\end{align*}
\begin{itemize}
\item $R^2 \in [0, 1]$ 
\end{itemize}
}

\frame{
\frametitle{What is $R^2$?}
\begin{align*}
R^2=\frac{\text{Explained Variation}}{\text{Total Variation}}
\end{align*}
\begin{itemize}
\item If Explained variation = Total Variation $\Rightarrow R^2=1$
\begin{itemize}
\item $R^2 = 1$ indicates that the model explains \textit{all the variablility}
\item $R^2 = 0$ indicates that the model explains \textit{none of the variability}
\end{itemize}
\item[$\Rightarrow$]The higher $R^2$, the better the model fits the observed data. 
\end{itemize}
}

\frame{
\frametitle{Why could $R^2$ become problematic?}
\begin{itemize}
\item $R^2$ only describes how well the model fits the observations, it does neither validate nor reject it
\item $R^2$ increase: Additional regressors always increase $R^2$, independent of their relevance
 \begin{align*}
 \text{assume we get } &R^2_1 \text{ from $y=X_1b_{11} + \hat{\epsilon_1}$} \\
 \text{and we get } &R^2_2 \text{ from $y=X_1b_{11}+X_2b_{22} + \hat{\epsilon_2}$}\\ \\
 \text{then } &R^2_2 \geq R^2_1 \text{ always holds}
 \end{align*}
\end{itemize}
}

\frame{
\frametitle{Proof: $R^2$ increase}
\begin{itemize}
\item Consider the sum of squared residuals 
\begin{align*}
S(b^*_{21}, b^*_{22}) = (y-X_1b^*_{21}+X_2b^*_{22})'(y-X_1b^*_{21}+X_2b^*_{22})
\end{align*}
\item This sum is minimized by OLS estimators $b_{21}$ and $b_{22}$:
\begin{align*}
\hat{\epsilon_2}'\hat{\epsilon_2} = S(b_{21}, b_{22}) \leq S(b_{11}, 0) = \hat{\epsilon_1}'\hat{\epsilon_1}
\end{align*}
\item This implies that 
\begin{align*}
1-\frac{\sum_{i=1}^{n}\hat{\epsilon_2}'\hat{\epsilon_2}}{\sum_{i=1}^{n}(y_i-\bar{y})^2} &\geq 1-\frac{\sum_{i=1}^{n}\hat{\epsilon_1}'\hat{\epsilon_1}}{\sum_{i=1}^{n}(y_i-\bar{y})^2} \\\\
\Leftrightarrow R_2^2 &\geq R_1^2 \qed
\end{align*} 
\end{itemize}
}

\frame{
\frametitle{$R^2$ adjusted}
\begin{itemize}
\item Introduce an adjusted $\bar{R}^2$ to deal with this problem. 
\begin{align*}
\bar{R}^2 &= 1- \frac{\frac{1}{n-K}\sum_{i=1}^n \hat{\epsilon_i}^2}{\frac{1}{n-1}\sum_{i=1}^{n}(y_i-\bar{y})^2} \\
&= 1 - \frac{n-1}{n-K}(1-R^2)
\end{align*}
\item $\bar{R}^2$ is better than $R^2$ if $\bar{R}^2 \leq R^2$
\end{itemize}
}

\frame{
\frametitle{Proof: $\bar{R}^2 \leq R^2$}
\begin{align*}
\bar{R}^2 &= 1 - \frac{n-1}{n-K}(1-R^2)\\
&= 1 - \frac{n-1}{n-K} +\frac{n-1}{n-K}R^2 -\underbrace{\frac{K-1}{n-K}R^2 + \frac{K-1}{n-K}R^2}_{=0} \\
&= 1 - \frac{n-1}{n-K} + R^2 + \frac{K-1}{n-K}R^2\\[1em]
&= -\frac{K-1}{n-K} + R^2 + \frac{K-1}{n-K}R^2 \\[1em]
&= R^2 -\frac{K-1}{n-K}(1-R^2) \leq R^2 \qed
\end{align*}
}

\frame{
\frametitle{Conclusion}
\begin{itemize}
\item $R^2$ shows how well a model fits the observed data 
\item but: $R^2$ increases with the number of regressors eventhough they might not be relevant for the model
\item therefore we need to adjust it and use $\bar{R}^2$
\end{itemize}
}
\end{document}